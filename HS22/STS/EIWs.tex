\documentclass[STS.tex]{subfiles}

\usepackage{amsmath}

\begin{document}

\section{Elementare Wahrscheinlichkeitsrechnung}


\begin{tabular}{|c|c|c|c|}
\hline
 Eigenschaft & Symbol & Formel & Beschreibung \\
\hline
 Ergebnisraum & $ \Omega $ & & \\
\hline

  Zieldichte &
  $ \rho $ &
  $ \rho : \Omega \rightarrow [ 0,1 ] $ &
  $ \sum_{\omega \in \Omega} \rho ( \omega )=1 $ \\
\hline

  Ereignis &
  $ A $ &
  Teilmenge von $ \Omega $ &
  Leere Menge $ {} $ entspricht dem unmöglichen Ereignis \\
\hline

Ereignisraum &
$2^{ \Omega } $ &
&
Menge aller möglichen Ereigniss. $ \Omega $ \\
\hline

  Wahrscheinlichkeitsmass &
  $ P $ &
  $ P: 2^{ \Omega } \rightarrow [0,1] $ &
  $ sum_{\omega \in M} \rho ( \omega )=1, M \subseteq \Omega $ \\
\hline
\end{tabular}


\subsection{Kenngrössen}

\textbf{Mittelwert}: $ \mu = \frac{m}{n}$


\end{document}
