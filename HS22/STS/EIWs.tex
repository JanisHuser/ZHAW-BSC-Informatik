\documentclass[STS.tex]{subfiles}

\usepackage{amsmath}
\usepackage{tikz}
\usetikzlibrary{trees}

\begin{document}

\section{Elementare Wahrscheinlichkeitsrechnung}


\begin{tabular}{|c|c|c|c|}
\hline
 Eigenschaft & Symbol & Formel & Beschreibung \\
\hline
 Ergebnisraum & $ \Omega $ & & \\
\hline

  Zieldichte &
  $ \rho $ &
  $ \rho : \Omega \rightarrow [ 0,1 ] $ &
  $ \sum_{\omega \in \Omega} \rho ( \omega )=1 $ \\
\hline

  Ereignis &
  $ A $ &
  Teilmenge von $ \Omega $ &
  Leere Menge $ {} $ entspricht dem unmöglichen Ereignis \\
\hline

Ereignisraum &
$2^{ \Omega } $ &
&
Menge aller möglichen Ereigniss. $ \Omega $ \\
\hline

  Wahrscheinlichkeitsmass &
  $ P $ &
  $ P: 2^{ \Omega } \rightarrow [0,1] $ &
  $ sum_{\omega \in M} \rho ( \omega )=1, M \subseteq \Omega $ \\
\hline
\end{tabular}


\subsection{Kenngrössen}

\textbf{Mittelwert}: $ \mu = \frac{m}{n}$


\subsection{Bedingte Wahrscheinlichkeiten}

Wahrscheinlichkeit für das Eintreten des Eeignisses B unter der Bedingung oder Voraussetzung, dass das Ereignis A eintreitt.

$ P(B|A) $

\subsubsection{Wahrscheinlichkeitsbaum}

\begin{itemize}
    \item {Alle von einem Blatt (Verzweigungspunkt) aus mit Pfeilen erreichbare Ereignisse sind paarweise disjunkt (d.h. sie schliessen sich gegenseitig aus)}
    \item{Die Summe der Übergangswahrscheinlichkeiten aller von einem Blatt ausgehenden Pfeile ist 1.}
\end{itemize}

\tikzstyle{level 1}=[level distance=1.5cm, sibling distance=3.5cm]
\tikzstyle{level 2}=[level distance=1cm, sibling distance=2cm]
\begin{tikzpicture}
\node[end]{}
  child {
    node[bag] {A}
    child {
      node [bag] {$Z$}
      edge from parent
      node[left] {0.9}
    }
    child {
    node [bag] {$\overline{Z}$}
      edge from parent
      node[right] {0.1}
    }
    edge from parent
    node[left] {0.5}
  }
  child {
    node[bag] {B}
    child {
      node [bag] {$Z$}
      edge from parent
      node[left] {0.7}
    }
    child {
    node [bag] {$\overline{Z}$}
      edge from parent
      node[right] {0.1}
    }
    edge from parent
    node[left] {0.3}
  }
  child {
    node[bag] {C}
    child {
      node [bag] {$Z$}
      edge from parent
      node[left] {0.4}
    }
    child {
    node [bag] {$\overline{Z}$}
      edge from parent
      node[right] {0.6}
    }
    edge from parent
    node[right] {0.2}
  }
\end{tikzpicture}

$
P(A \cup Z) = P(A) * P(Z|A)
=
0.5 * 0.9 = 0.45
$

\textbf{Allgemein gilt}: Wahrscheinlichkeiten längs eines Pfades werden multipliziert.


\textbf{Multiplikationssatz}:
$
P(A \cup B) = P(A) * P(B|A) = P(B) * P(A|B)$

\textbf{Satz von der Totalen Wahrscheinlichkeit}:
$
P(B)=P(A)*P(B|A)+P( \overline{A}) * P(B| \overline{A})
$

\textbf{Satz von Bayes}: $ P(B|A) = \frac{P (A \cap B)}{P (A) }$


\subsection{Stochastische Unabhängigkeit}
Wenn A und B stochastisch unabhängig sind, beeinflusst das Eintreten des inene Ereignisses also das Eintreten des anderen Ereignisses nicht.

\end{document}
