\documentclass[STS.tex]{subfiles}

\usepackage{amsmath}
\usepackage{tikz}
\usetikzlibrary{trees}

\begin{document}


\section{Diskrete und stetige Verteilung}

\textbf{Zufallsexperiment}: Ein Vorgang, bei dem folgende Bedingungen erfllt sind:
\begin{itemize}
    \item{Vorgäng lässt sich unter den gleichen äusseren Bedingungen beliebig oft wiederholen}
    \item{Mehrere sich gegenseitig ausschliesende Ergebnisse möglich}
    \item{Ergebnis lässt sich nicht mit Sicherheit voraussagen}
\end{itemize}


Mögliche, sich gegenseitig ausschliessende Ergebnisse $ \omega_1, \omega_2,... $ werden zur Menge $ \Omega $

\textbf{diskret}: ist immer numerisch \\
\textbf{stetig}: Sind numerische Werte, die zwischen zwei beliebigen Werten eine unendliche Anzahl von Werten aufweisen \\

\subsection{Disrekte und stetige Zufahlsvariablen}

\textbf{Kumulative Verteilungsfunktion}: $ F(x) = P (X \leq x)$

\subsection{Für stetige Zufallsvariablen}

$ F(x) = P(X \leq x)
=
 \int_{- \infty }^{x} f(u) du
$
\quad
f ist die Dichtefunktion / PDF \\

Es gilt immer: $ f(x) \geq 0 $ für alle $ x \in \mathbb{R} $
UND
$ \int_{- \infty }^{ \infty} f(u) du = 1
$

\textbf{Erwartungswert von X}:
$
\mu = E(X) = \int{ -\infty}^{ \infty} f(x) * x dx
$

\textbf{Varianz von X}:
$
\sigma^2 = V(X) = \int{- \infty}^{\infty} f(x) * (x- E(X))^2 dx
$

\textbf{Standardabweichung}:
$
\sigma = \sqrt{V(X)}
$


\end{document}