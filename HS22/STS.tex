\documentclass{article}

\begin{document}
	\title{STS}
	
	\section{Merkmale}
	
	\subsection{Qualitativ / Kategoriell}
	Es wird eine Auspr\"agung und kein Ausmass angegeben. Insbesondere gibt es nur endlich viele Auspr\"agungen.
	\subsubsection{Nominal}
	Keine Kategorisierung, keine Ordnung
	
	\subsubsection{Ordinal}
	Ordnung vorhanden, Rangierung möglich
	
	\subsection{Quantitativ / Metrisch}
	
	Es wird ein Ausmass angegeben. Das Ausmass wird mit Zahlen angegeben.
	
	\subsubsection{Quantitativ}
	Diskret endlich viele aber abz\a"hlbar unendlich viele Auspr\a"gunen.
	
	\subsubsection{Stetig}
	Alle Ausprägungen in einem reelen Interval


	\section{H\"aufigkeiten}
	
	\subsection{Graphische Darstellung}
	
	\textbf{Kategoriell}: Säulendiagramm, Skalardiagramm \\
	\textbf{Metrisch}: Säulendiagramm, Skalardiaramm, Histogramm bei Klassenbildung \\
	


\begin{tabular}{|c c c|} 
 \hline
 Beschreibung & Zeichen & Formel\\
 \hline
	
		Anzahl &
		$ n $ &
		\\ 
 	\hline
	
	
		Wert / Klasse &
		$ a_i $  &
		\\
 	\hline
 
 		Absolute H\"aufigkeit &
 		$ h_i $ &
 		Anzahl
 		\\
 	\hline

		Relative Häufigkeit (PMF) &
		$ f_i $ &
		$ \frac{h_i}{n} $ 
		\\
	\hline
	
 		Kummulative rel. H\"aufigkeit (CDF) & 
 		$F_i$ &
 		Summe aller vorgehenden $f_i$
 		\\
	\hline
	
		Spaltenbreite &
		$ d_i $ & 
		\\
	\hline
	
		Spaltenhöhe &
		$ h $ &
		$ \frac{h_i}{d_i} $
		\\
	\hline
	
		PDF-Wert &
		$f$ &
		$ \frac{f_i}{d_i} $
		\\
	\hline
	
\end{tabular}

	

\end{document}