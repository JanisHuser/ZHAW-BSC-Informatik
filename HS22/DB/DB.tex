\documentclass{article}

\usepackage{geometry}
\geometry{legalpaper, margin=0.5in}

\usepackage{xcolor}

\begin{document}

\section{SQL}

\subsection{Probleme}
1. Auswahl im Select, wo ColumnName mit ColumnName direkt verglichen wird

\begin{verbatim}{sql}
      SELECT *
FROM Besucher
WHERE EXISTS (SELECT 1
  FROM Besucher
  WHERE Vorname=Vorname AND
  NOT (Name=Name));

  \end{verbatim}





\subsection{Subqueries (in WHERE) mit EXISTS}


\begin{verbatim}{sql}
SELECT *
FROM Besucher AS x
WHERE EXISTS (SELECT 1
  FROM Besucher AS y
  WHERE x.Vorname=y.Vorname AND
  NOT (x.Name=y.Name));
\end{verbatim}


\subsection{Joins}
\subsubsection{Equijoin}
\color{red}Kreuzprodukt + Selektionsprädikat für gemeinsame Attrivute + autom. Umbenennung/Projektion von Attributen\color{black}


\begin{verbatim}
SELECT R.A, R.B FROM R, S WHERE R.B = S.B;  

SELECT R.A, R.B FROM R CROSS JOIN S WHERE R.B = S.B;

SELECT A,B FROM R NATURAL JOIN S;

SELECT R.A, R.B FROM R JOIN S ON R.B = S.B;
\end{verbatim}


\section{Aggregatfuntionen ohne Gruppierung}

\textbf{COUNT}: z\"ahlt die Anzahl der Elemente \\
\textbf{MAX}: maximaler Wert der Elemente \\
\textbf{MIN}: minimaler Wert der Elemente \\
\textbf{SUM}: Summe der Werte der Elemente \\
\textbf{AVG}: Durchschnitt der Werte der Elemente \\

SUM und AVG m\"ussen z\"ahlbar sein

\subsection{COUNT}

COUNT(<AttributName>) zählt nur diejenigen Tupel, bei denen der Wert des Attributs nicht NULL ist. \\
COUNT(*) zählt alle Tupel, es gibt kein Tupel, bei dem alle Attribute gleichzeigit NULL sein k\"onnen. \\
COUNT(\color{blue} DISTINCT \color{black}NAME): zählt die Anzahl verschiedener Namen

\subsection{Bestellumsatz pro Kunde, unabhängig von der einzelnen Bestellung}

\begin{verbatim}
SELECT kdNR, SUM(menge * preis) AS UMSATZ
FROM Bestellposition JOIN Kaufhistorei ON bestNr = bNr
GROUP BY kdNr;
\end{verbatim}

\subsection{Grösster Bestellumsatz}

\begin{verbatim}
  SELECT MAX(Umsatz) AS GroessterUmsatz
  FROM (SELECT kdNR, SUM(menge * preis) AS UMSATZ
  FROM Bestellposition JOIN Kaufhistorei ON bestNr = bNr
  GROUP BY kdNr) AS x;
  \end{verbatim}


\section{Aggregatsfunktionen mit Gruppierung}

\subsection{Reihenfolge der SQL Abfragen}
1. FROM
2. WHERE
3. GORUP BY
4. HAVING
5. SELECT
6. ORDER BY


\subsubsection{HAVING}
HAVING COUNT(*) > 2: Alle Sätze, wo alle numerischen Werte grösser als 2 sind.


\end{document}